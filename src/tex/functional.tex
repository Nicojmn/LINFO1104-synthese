\part{Functional programming}
\chapter{Introduction à la programmation fonctionnelle}
\minitoc
\newpage

\section{Caractérisques}

    \paragraph{}C'est le premier paradigme que nous allons voir. Voici ses caractéristiques :

        \begin{tbox}{info} % infos de la slide 13
            \begin{itemize}[label=\textbullet, font=\small]
                \item Paradigme le plus simple
                \item Fondement de {\color{success} tous les autres paradigmes}
                \item C'est une forme de declarative programming : \textit{say what, not how}
            \end{itemize}
            \tcblower
            \begin{itemize}[label=\textbullet, font=\small]
                \item Introduction pour les concepts de programmation
                \item Introduction au langage noyau
                \item Sert à expliquer l'{\color{danger}invariant programming}
                \item Sert à expliquer la {\color{danger}symbolic programming}
                \item Sert à expliquer la {\color{danger}higher-order programming}
                \item {\color{danger} Sémantique formelle} basée sur le langage noyau
            \end{itemize}
        \end{tbox}